
%package list
\documentclass{article}
\usepackage[top=3cm, bottom=3cm, outer=3cm, inner=3cm]{geometry}
\usepackage{graphicx}
\usepackage{url}
%\usepackage{cite}
\usepackage{hyperref}
\usepackage{array}
\usepackage{multicol}
\newcolumntype{x}[1]{>{\centering\arraybackslash\hspace{0pt}}p{#1}}
\usepackage{natbib}
\usepackage{pdfpages}
\usepackage{multirow}
\usepackage{float}
\usepackage[normalem]{ulem}
\useunder{\uline}{\ul}{}
\usepackage{svg}
\usepackage{amsmath}
\usepackage{hyperref}

%%%%%%%%%%%%%%%%%%%%%%%%%%%%%%%%%%%%%%%%%%%%%%%%%%%%%%%%%%%%%%%%%%%%%%%%%%%%
%%%%%%%%%%%%%%%%%%%%%%%%%%%%%%%%%%%%%%%%%%%%%%%%%%%%%%%%%%%%%%%%%%%%%%%%%%%%
\newcommand{\csemail}{vmachacaa@unsa.edu.pe}
\newcommand{\csdocente}{Vicente Machaca Arceda}
\newcommand{\cscurso}{Algoritmos y Estructura de Datos}
\newcommand{\csuniversidad}{Universidad Nacional de San Agustín}
\newcommand{\csescuela}{Maestría en Ciencias de la Computación}
\newcommand{\cspracnr}{04}
\newcommand{\cstema}{Kd-tree}
%%%%%%%%%%%%%%%%%%%%%%%%%%%%%%%%%%%%%%%%%%%%%%%%%%%%%%%%%%%%%%%%%%%%%%%%%%%%
%%%%%%%%%%%%%%%%%%%%%%%%%%%%%%%%%%%%%%%%%%%%%%%%%%%%%%%%%%%%%%%%%%%%%%%%%%%%


\usepackage[english,spanish]{babel}
\usepackage[utf8]{inputenc}
\AtBeginDocument{\selectlanguage{spanish}}
\renewcommand{\figurename}{Figura}
\renewcommand{\refname}{Referencias}
\renewcommand{\tablename}{Tabla} %esto no funciona cuando se usa babel
\AtBeginDocument{%
	\renewcommand\tablename{Tabla}
}

\usepackage{fancyhdr}
\pagestyle{fancy}
\fancyhf{}
\setlength{\headheight}{30pt}
\renewcommand{\headrulewidth}{1pt}
\renewcommand{\footrulewidth}{1pt}
\fancyhead[L]{\raisebox{-0.2\height}{\includegraphics[width=3cm]{img/logo_unsa}}}
\fancyhead[C]{}
\fancyhead[R]{\fontsize{7}{7}\selectfont	\csuniversidad \\ \csescuela \\ \textbf{\cscurso} }
\fancyfoot[L]{Grupo N◦ 02}
\fancyfoot[C]{\cscurso}
\fancyfoot[R]{Página \thepage}

\begin{document}

\vspace*{10px}

\begin{center}
	\fontsize{17}{17} \textbf{ Práctica \cspracnr}
\end{center}

%\centerline{\textbf{\underline{\Large Título: Informe de revisión del estado del arte}}}
%\vspace*{0.5cm}

\begin{table}[h]
	\begin{tabular}{|x{4.7cm}|x{4.8cm}|x{4.8cm}|}
		\hline
		\textbf{DOCENTE} & \textbf{CARRERA} & \textbf{CURSO} \\
		\hline
		\csdocente       & \csescuela       & \cscurso       \\
		\hline
	\end{tabular}
\end{table}

\begin{table}[h]
	\begin{tabular}{|x{4.7cm}|x{4.8cm}|x{4.8cm}|}
		\hline
		\textbf{PRÁCTICA} & \textbf{TEMA} & \textbf{DURACIÓN} \\
		\hline
		\cspracnr         & \cstema       & --                \\
		\hline
	\end{tabular}
\end{table}

\section{Integrantes}
\begin{itemize}
	\item Grupo N° 2
	\item Integrantes:
	      \begin{itemize}
		      \item EDER ALONSO AMPUERO ATAMARI
		      \item HOWARD FERNANDO ARANZAMENDI MORALES
		      \item JOSE EDISON PEREZ MAMANI
		      \item HENRRY IVAN ARIAS MAMANI
	      \end{itemize}
\end{itemize}

\section{Repositorio GitHub}
URL Github: \href{https://github.com/hAriasm/Practica4_ayed}{Repositorio Práctica 4 AyED}

\section{Marco Teórico}
\subsection{Kd-Tree}

\subsubsection{Definición}
\paragraph{}
Un KD-Tree (también llamado árbol K-dimensional) es un árbol de búsqueda binaria donde los datos en cada nodo son un punto K-dimensional en el espacio. En resumen, es una estructura de datos de partición de espacio (detalles a continuación) para organizar puntos en un espacio K-Dimensional.

Un nodo que no es una hoja en el árbol K-D divide el espacio en dos partes, llamadas medios espacios.

Los puntos a la izquierda de este espacio están representados por el subárbol izquierdo de ese nodo y los puntos a la derecha del espacio están representados por el subárbol derecho. Pronto estaremos explicando el concepto de cómo se divide el espacio y se forma el árbol.

En aras de la simplicidad, entendamos un árbol 2-D con un ejemplo.

La raíz tendría un plano alineado con el eje x, los hijos de la raíz tendrían ambos planos alineados con el eje y, los nietos de la raíz tendrían todos planos alineados con el eje x, y los bisnietos de la raíz tendrían todos planos alineados con el eje y, y así sucesivamente.

\subsubsection{Generalización}
\paragraph{}
Numeremos los planos como 0, 1, 2, …(K – 1). Del ejemplo anterior, es bastante claro que un punto (nodo) en la profundidad D tendrá un plano alineado donde A se calcula como:

A = D mod K

\subsubsection{¿Cómo determinar si un punto estará en el subárbol izquierdo o en el subárbol derecho?}
\paragraph{}
Si el nodo raíz esta alineado en el plano A, el subárbol izquierdo contendrá todos los puntos cuyas coordenadas en ese plano sean más pequeñas que las del nodo ráiz. De manera similar, el subárbol derecho contendrá todos los puntos cuyas coordenadas en ese plano sean mayores-iguales que las del nodo raíz.

\subsubsection{Creación de un árbol 2-D}
\paragraph{}
Considere los siguientes puntos en un plano 2-D:
(3, 6), (17, 15), (13, 15), (6, 12), (9, 1), (2, 7), (10, 19)

\begin{enumerate}
	\item Insertar (3, 6): dado que el árbol está vacío, conviértalo en el nodo raíz.
	\item Insertar (17, 15): compararlo con el punto del nodo raíz. Dado que el nodo raíz está alineado con X, el valor de la coordenada X se comparará para determinar si se encuentra en el subárbol derecho o en el subárbol izquierdo. Este punto estará alineado con Y.
	\item Insertar (13, 15): el valor X de este punto es mayor que el valor X del punto en el nodo raíz. Entonces, esto estará en el subárbol derecho de (3, 6). Nuevamente compare el valor Y de este punto con el valor Y del punto (17, 15) (¿Por qué?). Como son iguales, este punto estará en el subárbol derecho de (17, 15). Este punto estará alineado con X.
	\item Insertar (6, 12): el valor X de este punto es mayor que el valor X del punto en el nodo raíz. Entonces, esto estará en el subárbol derecho de (3, 6). Nuevamente compare el valor Y de este punto con el valor Y del punto (17, 15) (¿Por qué?). Como 12 < 15, este punto estará en el subárbol izquierdo de (17, 15). Este punto estará alineado con X.
	\item Insertar (9, 1): De manera similar, este punto estará a la derecha de (6, 12).
	\item Insertar (2, 7): De manera similar, este punto estará a la izquierda de (3, 6).
	\item Inserta (10, 19): De manera similar, este punto estará a la izquierda de (13, 15).
\end{enumerate}

\begin{figure}[h!]
	\centering
	\includegraphics[width=0.8\textwidth]{img/kdtree_01.png}
	\caption{Nodos en KD-TREE}
	\label{fig:kdtree_01}
\end{figure}

\subsubsection{¿Cómo se divide el espacio?}
\paragraph{}
Los 7 puntos se trazarán en el plano X-Y de la siguiente manera:

\begin{enumerate}
	\item El punto (3, 6) dividirá el espacio en dos partes: Dibujar la línea X = 3.
	\item El punto (2, 7) dividirá el espacio a la izquierda de la línea X = 3 en dos partes horizontalmente. Dibuja la línea Y = 7 a la izquierda de la línea X = 3.
	\item El punto (17, 15) dividirá el espacio a la derecha de la línea X = 3 en dos partes horizontalmente. Dibuja la línea Y = 15 a la derecha de la línea X = 3.
	\item El punto (6, 12) dividirá el espacio debajo de la línea Y = 15 ya la derecha de la línea X = 3 en dos partes. Dibuja la línea X = 6 a la derecha de la línea X = 3 y debajo de la línea Y = 15.
	\item El punto (13, 15) dividirá el espacio debajo de la línea Y = 15 ya la derecha de la línea X = 6 en dos partes. Dibuja la línea X = 13 a la derecha de la línea X = 6 y debajo de la línea Y = 15.
	\item El punto (9, 1) dividirá el espacio entre las líneas X = 3, X = 6 e Y = 15 en dos partes. Dibuja la línea Y = 1 entre las líneas X = 3 y X = 13.
	\item El punto (10, 19) dividirá el espacio a la derecha de la línea X = 3 y arriba de la línea Y = 15 en dos partes. Dibuja la línea Y = 19 a la derecha de la línea X = 3 y arriba de la línea Y = 15.
\end{enumerate}

\begin{figure}[h!]
	\centering
	\includegraphics[width=0.8\textwidth]{img/kdtree_02.png}
	\caption{División del espacio en KD-TREE}
	\label{fig:kdtree_02}
\end{figure}

\subsection{KNN}
\subsubsection*{Definición}
\paragraph{}
El algoritmo de k-vecinos más cercanos, también conocido como KNN o k-NN, es un clasificador de aprendizaje supervisado no paramétrico, que utiliza la proximidad para hacer clasificaciones o predicciones sobre la agrupación de un punto de datos individual. Si bien se puede usar para problemas de regresión o clasificación, generalmente se usa como un algoritmo de clasificación, partiendo de la suposición de que se pueden encontrar puntos similares cerca uno del otro.


\section{Ejercicios}
\subsection{Ejercicio N° 1}
\subsection{Ejercicio N° 2}
\subsection{Ejercicio N° 3}
\subsection{Ejercicio N° 4}
\subsection{Ejercicio N° 5}
\subsection{Ejercicio N° 6}
\subsection{Ejercicio N° 7}
\subsection{Ejercicio N° 8}
\subsection{Ejercicio N° 9}
\subsection{Ejercicio N° 10}
\vspace{5mm}

\clearpage

\section{Conclusiones}
\begin{itemize}
	\item El quadtree se codifican en un árbol de cuatro hijos no balanceado, y el  octree se codifica en un árbol octal no balanceado.
	\item De la bibliografía revisada, se concluye que las aplicaciones más comunes para los QuadTree y OcTree son el Procesamiento de imágenes, generación de mallas e indexado espacial.
	\item Las estructuras octree son usadas mayormente para partir un espacio tridimensional, dividiéndolo recursivamente en ocho octantes, siendo análogas tridimensionales de los quadtree bidimensionales.
\end{itemize}

\clearpage
\section{Referencias}
\begin{enumerate}
	\item Tobler, W., & Chen, Z. T. (1986). A quadtree for global information storage. Geographical Analysis, 18(4), 360-371.

	\item Samet, H. (1984). The quadtree and related hierarchical data structures. ACM Computing Surveys (CSUR), 16(2), 187-260.

	\item Rojas Delgado, J. (2015). Estructura de datos espacial jerárquica para la indexación de agrupaciones de objetos geométricos (Bachelor's thesis, Universidad de las Ciencias Informáticas. Facultad 6).

	\item Castillo Oliva, A. D. (2018). Visualización de grandes volúmenes de datos empleando Octree (Bachelor's thesis, Universidad de las Ciencias Informáticas. Centro Vertex, Entornos Interactivos 3D. Facultad 4).

	\item Rivero, J. P. S., de la Hoz, Á. P., & Medina, M. Á. P. GENERACIÓN DE MALLAS APLICACIONES A LA INGENIERÍA.

	\item Brunet Crosa, P., Santisteve i Puyuelo, F., Vilanova, A., Chiarabini, L., Patow, G. A., Staffetti, E., & Surinyac, J. (1999). Estructuras geométricas jerárquicas para la modelización de escenas 3D.

	\item \href{https://p5js.org/es/libraries/}{P5.js}
	\item \href{https://www.geeksforgeeks.org/}{Geeksforgeeks}

\end{enumerate}

\end{document}
